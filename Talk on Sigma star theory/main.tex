\documentclass[12pt,a4paper]{article}

% Packages
\usepackage[utf8]{inputenc} 
\usepackage{amssymb}
\usepackage{faktor} 
\usepackage{fancyhdr}
\usepackage{todonotes}
\usepackage{amsmath}
\usepackage{hyperref}
%\usepackage[capitalize,nameinlink]{cleveref}
\usepackage{amsthm}
% \usepackage[backend=bibtex,style=verbose-trad2]{biblatex}
% \usepackage[printsolution=true]{exercises}

% Theorem Environments
\theoremstyle{nicestyle}
\newtheorem{theorem}{Theorem}[subsection]
\providecommand*{\theoremautorefname}{Theorem}
\newtheorem{problem}{Problem}[subsection]
\providecommand*{\problemeautorefname}{Problem}
\newtheorem{hint}{Hint}[subsection]
\providecommand*{\hintautorefname}{Problem}
\newtheorem{exercise}{Exercise}[subsection]
\providecommand*{\exerciseautorefname}{Exercise}
\newtheorem{definition}{Definition}[subsection]
\providecommand*{\definitionautorefname}{Definition}
\newtheorem{lemma}{Lemma}[subsection]
\providecommand*{\lemmaautorefname}{Lemma}
\newtheorem{proposition}{Proposition}[subsection]
\providecommand*{\propositionautorefname}{Propositon}
\newtheorem{corollary}{Corollary}[subsection]
\providecommand*{\corollaryautorefname}{Corollary}
\newtheorem{claim}{Claim}[subsection]
\providecommand*{\claimautorefname}{Claim}
\newtheorem{subclaim}{Subclaim}[subsection]
\providecommand*{\subclaimautorefname}{Subclaim}
\newtheorem{convention}{Convention}[subsection]
\providecommand*{\conventionautorefname}{Convention}
\newtheorem{remark}{Remark}[subsection]
\providecommand*{\remarkautorefname}{Remark}
\newtheorem{fact}{Fact}[subsection]
\providecommand*{\factautorefname}{Fact}
\newtheorem{example}{Example}[subsection]
\providecommand*{\exampleautorefname}{Example}
\newtheorem{notation}{Notation}[subsection]
\providecommand*{\notationautorefname}{Notation}
\newtheorem{question}{Question}[subsection]
\providecommand*{\questionautorefname}{Question}

\newtheorem*{exercise*}{Exercise}
\newtheorem*{theorem*}{Theorem}
\newtheorem*{lemma*}{Lemma}
\newtheorem*{proposition*}{Proposition}
\newtheorem*{corollary*}{Corollary}
\newtheorem*{claim*}{Claim} 
\newtheorem*{subclaim*}{Subclaim}
\newtheorem*{convention*}{Convention}

% Proofblack
\newenvironment{proofblack}{\begin{proof}}
  {\renewcommand{\qedsymbol}{$\blacksquare$}\end{proof}}

% Math Operators
\DeclareMathOperator{\card}{card}
\DeclareMathOperator{\col}{Col}
\DeclareMathOperator{\dom}{dom}
\DeclareMathOperator{\ran}{ran}
\DeclareMathOperator{\rank}{rank} 
\DeclareMathOperator{\supp}{supp}
\DeclareMathOperator{\ord}{Ord}
\DeclareMathOperator{\limit}{Lim} 
\DeclareMathOperator{\zfc}{ZFC}
\DeclareMathOperator{\zf}{ZF} 
\DeclareMathOperator{\tc}{tc}
\DeclareMathOperator{\rk}{rk} 
\DeclareMathOperator{\cf}{cf}
\DeclareMathOperator{\id}{id}
\DeclareMathOperator{\fn}{Fn}
\DeclareMathOperator{\ult}{Ult}
\DeclareMathOperator{\rS}{r \Sigma}
\DeclareMathOperator{\crit}{crit}
\DeclareMathOperator{\trcl}{trcl}
\DeclareMathOperator{\lh}{lh}
\DeclareMathOperator{\wfp}{wfp}
\DeclareMathOperator{\hull}{Hull}
\DeclareMathOperator{\otp}{otp}
\DeclareMathOperator{\pr}{pr}
\DeclareMathOperator{\lex}{lex}
\DeclareMathOperator{\length}{lh}
\DeclareMathOperator{\gch}{GCH}
\DeclareMathOperator{\rud}{rud}

\begin{document}
\author{Stefan Mesken}
\title{Iterated Reducts and $\Sigma^{*}$-Theory \\
  Exercises}
\maketitle

Let $M = (J_{\alpha}^{A}; \in, A, B)$ be an acceptable
(which also means amenable) $\mathcal{J}$-structure.

In today's seminar, we've introduced
\begin{enumerate}
\item the iterated projecta $(\omega\rho^{n}_{M} \mid n < \omega)$ (section 1.5 in Zeman's book),
\item the set of $n$-good parameters $P^{n}_{M}$ (section 1.5 in Zeman's book),
\item the $n$-th master code $A^{n,p}_{M}$ for
  $p \in P^{n}_{M}$ (section 1.5 in Zeman's book),
\item the $n$-th reduct $M^{n,p}$ for $p \in P^{n}_{M}$ (section 1.5 in Zeman's book),
\item $\Sigma^{(n)}_{l}$-formulae (section 1.6 in Zeman's book),
\item the modeling relation $M \models \phi[\vec{a}]$ for
  $\vec{a} \in M$ and $\phi \in \Sigma^{(n)}_{l}$ (section 1.6 in
  Zeman's book).
\end{enumerate}

Here are some voluntary exercises of varying difficulty: \footnote{If
  you decide to tackle (some of) them, you're more than welcome to
  email me your solutions at
  \href{mailto:stefan.m@wwu.de}{stefan.m@wwu.de} for feedback or to
  discuss them with me after next week's seminar.}

\begin{exercise}
  Let $\kappa$ be an $M$-cardinal. Then for every
  $a \in H^{M}_{\kappa}$
  \[
    \mathcal{P}(a) \cap M \subseteq H_{\kappa}^{M}.
  \]
\end{exercise}

(Hint: $M$ is acceptable.)

\begin{exercise}
  Find a non-acceptable $N = (J^{C}_{\gamma}; \in, C, D)$ such that
  the claim above fails.
\end{exercise}

(Hint: If you are already familiar with basic inner model theory,
consider $L[U]$ for a normal ultrafilter $U \in V$ on $\kappa$ and
compute at which level of $L[U]$ $0^{\#}$ (coded as a subset of
$\omega$) appears. \\
Alternatively, add a ``delayed Cohen real'' to $L$ via forcing,
i.e. consider an isomorphic copy $\mathbb{P}$ of Cohen forcing in
$L \setminus J_{\omega_{1}^{L}}$, let $g \subseteq \mathbb{P}$ be
generic and consider $(J^{g}_{\kappa}; \in, g)$ for some sufficiently
large $L$-cardinal $\kappa$.)


\begin{exercise}
  Let $p \in P^{1}_{M}$. Then
  \begin{enumerate}
  \item There is some set $A^{+}$ which is $\Sigma_{1}(M)$ in the
    parameter $p(0)$ such that
    \[
      A^{1,p}_{M} = A^{+} \cap (\omega \times
      H^{M}_{\omega\rho^{1}_{M}}).
    \]
    (And there is in fact a $\Sigma_{1}$-definition that defines
    $A^{1,p}_{M}$ in this fashion uniformly for all acceptable
    $\mathcal{J}$-structures of the same signature as $M$
    \footnote{i.e. every acceptable $\mathcal{J}$-structure $N$ of the
      form $N = (J^{C}_{\gamma}; \in, C, D)$, with $1$-ary predicates $C,D$}.)
  \item For every $p \in P^{1}_{M}$
    \[
      M^{1,p} := (H^{M}_{\omega\rho^{1}_{M}}; \in, A^{1,p}_{M})
    \]
    is an acceptable $\mathcal{J}$-structure.
  \end{enumerate}
\end{exercise}

(Hint: Let $(\phi_{i} \mid i < \omega)$ be a recursive enumeration of all $\Sigma_{1}$-formulae in the language of $M$. Recall that
  \[
    \models^{M}_{\Sigma_{1}} := \{ (i, x) \in \omega \times M \mid M \models \phi_{i}[x] \}
  \]
  is uniformly $\Sigma_{1}$-definable over $M$.  )

  \begin{exercise}
    Show that $M^{n,p}$ is acceptable for all $n < \omega$ and all
    $p \in P^{n}_{M}$.
  \end{exercise}

  \begin{exercise}
    Suppose that $\omega\rho^{n+1}_{M} < o(M)$. Prove that
    \[
      M \models \omega\rho^{n+1}_{M} \text{ is a cardinal.}
    \]
  \end{exercise}

  (Hint: First prove that
  \[
    \omega\rho^{n+1}_{M} < \omega\rho^{n}_{M} \implies
    (H^{M}_{\omega\rho^{n}_{M}}; \in) \models \omega\rho^{n+1}_{M}
    \text{ is a cardinal}
  \]
  and then use acceptability.)

  \begin{exercise}
    Let $Q$ be a $\Sigma_{1}(M)$ definable in the parameter $p \in M$. Show that
    \[
      \Sigma_{l}(H^{M}_{\omega\rho^{1}_{M}}; \in, Q) \subseteq
      \Sigma^{(1)}_{l}(M) \text{ in parameter } p.
    \]
  \end{exercise}

  (Hint: $\Sigma^{(0)}_{1} \subseteq \Sigma^{(1)}_{l}$.)

\end{document}