\documentclass[12pt,a4paper]{article}

% Packages
\usepackage[utf8]{inputenc} 
\usepackage{amssymb}
\usepackage{faktor} 
\usepackage{fancyhdr}
\usepackage{todonotes}
\usepackage{amsmath}
\usepackage{hyperref}
%\usepackage[capitalize,nameinlink]{cleveref}
\usepackage{amsthm}
% \usepackage[backend=bibtex,style=verbose-trad2]{biblatex}
% \usepackage[printsolution=true]{exercises}


% Theorem Environments
\theoremstyle{nicestyle}
\newtheorem{theorem}{Theorem}[subsection]
\providecommand*{\theoremautorefname}{Theorem}
\newtheorem{problem}{Problem}[subsection]
\providecommand*{\problemeautorefname}{Problem}
\newtheorem{hint}{Hint}[subsection]
\providecommand*{\hintautorefname}{Problem}
\newtheorem{exercise}{Exercise}[subsection]
\providecommand*{\exerciseautorefname}{Exercise}
\newtheorem{definition}{Definition}[subsection]
\providecommand*{\definitionautorefname}{Definition}
\newtheorem{lemma}{Lemma}[subsection]
\providecommand*{\lemmaautorefname}{Lemma}
\newtheorem{proposition}{Proposition}[subsection]
\providecommand*{\propositionautorefname}{Propositon}
\newtheorem{corollary}{Corollary}[subsection]
\providecommand*{\corollaryautorefname}{Corollary}
\newtheorem{claim}{Claim}[subsection]
\providecommand*{\claimautorefname}{Claim}
\newtheorem{subclaim}{Subclaim}[subsection]
\providecommand*{\subclaimautorefname}{Subclaim}
\newtheorem{convention}{Convention}[subsection]
\providecommand*{\conventionautorefname}{Convention}
\newtheorem{remark}{Remark}[subsection]
\providecommand*{\remarkautorefname}{Remark}
\newtheorem{fact}{Fact}[subsection]
\providecommand*{\factautorefname}{Fact}
\newtheorem{example}{Example}[subsection]
\providecommand*{\exampleautorefname}{Example}
\newtheorem{notation}{Notation}[subsection]
\providecommand*{\notationautorefname}{Notation}
\newtheorem{question}{Question}[subsection]
\providecommand*{\questionautorefname}{Question}

\newtheorem*{exercise*}{Exercise}
\newtheorem*{theorem*}{Theorem}
\newtheorem*{lemma*}{Lemma}
\newtheorem*{proposition*}{Proposition}
\newtheorem*{corollary*}{Corollary}
\newtheorem*{claim*}{Claim} 
\newtheorem*{subclaim*}{Subclaim}
\newtheorem*{convention*}{Convention}

% Proofblack
\newenvironment{proofblack}{\begin{proof}}
  {\renewcommand{\qedsymbol}{$\blacksquare$}\end{proof}}

% Math Operators
\DeclareMathOperator{\card}{card}
\DeclareMathOperator{\col}{Col}
\DeclareMathOperator{\dom}{dom}
\DeclareMathOperator{\ran}{ran}
\DeclareMathOperator{\rank}{rank} 
\DeclareMathOperator{\supp}{supp}
\DeclareMathOperator{\ord}{Ord}
\DeclareMathOperator{\limit}{Lim} 
\DeclareMathOperator{\zfc}{ZFC}
\DeclareMathOperator{\zf}{ZF} 
\DeclareMathOperator{\tc}{tc}
\DeclareMathOperator{\rk}{rk} 
\DeclareMathOperator{\cf}{cf}
\DeclareMathOperator{\id}{id}
\DeclareMathOperator{\fn}{Fn}
\DeclareMathOperator{\ult}{Ult}
\DeclareMathOperator{\rS}{r \Sigma}
\DeclareMathOperator{\crit}{crit}
\DeclareMathOperator{\trcl}{trcl}
\DeclareMathOperator{\lh}{lh}
\DeclareMathOperator{\wfp}{wfp}
\DeclareMathOperator{\hull}{Hull}
\DeclareMathOperator{\otp}{otp}
\DeclareMathOperator{\pr}{pr}
\DeclareMathOperator{\lex}{lex}
\DeclareMathOperator{\length}{lh}
\DeclareMathOperator{\gch}{GCH}
\DeclareMathOperator{\rud}{rud}

\begin{document}
\author{Stefan Mesken}
\title{Fine Structure Seminar \\
Rutgers University}
\maketitle

\setcounter{section}{0}
\section{Fine Structure}

Unless specified otherwise, $M,N$ are acceptable
$\mathcal{J}$-structures, $k,l,m,n < \omega$.

\setcounter{subsection}{7}
\subsection{Substitution and Good Functions}

\todo[inline]{Talk 1, first draft}

\begin{problem}
  $\Sigma^{(n)}_1(M)$-relations are not necessarily closed under
  substitution of $\Sigma^{(n)}_1(M)$-functions.
\end{problem}

\begin{exercise}
  Find an example for a $\Sigma^{(n)}_1(M)$-relation $R$ and a
  (partial) $\Sigma^{(n)}_1(M)$-function $f$ such that $R \circ f$ is
  not a $\Sigma^{(n)}_1(M)$ relation.
\end{exercise}

\begin{hint}
  Suppose that $M$ is an acceptable $\mathcal{J}$-structure
  s.t. $\omega \rho^2_M < \omega \rho^1_M$ and such that there is some
  $\Sigma^{(0)}_0$-formula $\phi$ and some $p in M$ with
  \[
    A := \{ \xi < \omega \rho^2_M \mid M \models \exists x^1 \phi [
    \xi, x^{(1)}, p ] \} \not \in M.
  \]
  (Such $M$ exist. In fact, we can pick $M = J_{\alpha}$ for some countable $\alpha$.) \\
  Let $f$ be the partial $\Sigma^{(0)}_1(M)$-function defined by the
  formula
  \[
    \psi(y^{(0)}, z^{(1)}) : \iff y^{(0)} = z^{(1)}.
  \]
 Consider $\phi(x^{(0)}, f(y^{(0)}), p)$.
\end{hint}

Our goal is to identify a sufficiently large collection of
$\Sigma^{(n)}_1$-functions that can be substitued into
$\Sigma^{(n)}_1$-relations without increasing their complexity.

\begin{lemma}
  Let $n < \omega$ and $R(\vec{x}^{0}, \ldots, \vec{x}^n)$ be a
  $\Sigma^{(n)}_1(M)$-relation. Let $F^0, \ldots, F^{n}$ be such that
  for all $i \le n$ $F^i(\vec{x}^{0}, \ldots, \vec{x}^{(n)})$ is a
  partial $\Sigma^{(i)}_1(M)$-function to $H^i_{M}$. Then
  \[
    R(F^{0}(\vec{x}^{0}, \ldots, \vec{x}^{n}), \ldots,
    F^{n}(\vec{x}^{0}, \ldots, \vec{x}^n))
  \]
  is (uniformly) a $\Sigma^{(n)}_1(M)$-relation.
\end{lemma}

\begin{proof}
  By induction on $n$. The case $n=0$ is a simplification of the induction step
  $n \mapsto n+1$, hence we will only consider the latter: \\

  Let $R(\vec{x}^{0}, \ldots, \vec{x}^n)$ be a $\Sigma^{(n+1)}_1(M)$ relation and let $F^0, \ldots, F^{n+1}$ be as above. Write
  \[
    R(\vec{x}^{0}, \ldots, \vec{x}^{n+1}) \equiv \exists z^{n+1}
    B(\vec{v^{n+1}}, \vec{x}^{n+1}) R^{*}(\vec{x}^{0}, \ldots,
    \vec{x}^{n+1}, \vec{v}^{n+1}, z^{n+1})
  \]
  with
  \begin{enumerate}
  \item $B$ being a block of bounded quantifiers and
  \item $R^{*}$ being a Boolean combinatin of $\Sigma^{(n)}_1(M)$-relations.
  \end{enumerate}
  By our induction hypothesis and the fact that every $\Sigma^{(n)}_1(M)$-relation is the specialization of a $\Sigma^{(n)}_1(M)$ with arguments of type $\le n$, we obtain that
  \[
    R(F^{0}(\vec{y}), \ldots, F^{n}(\vec{y}), \vec{x}^{n+1})
  \]
  is $\Sigma^{(n+1)}_1(M)$. But now
  \[
    R(F^{0}(\vec{y}), \ldots, F^{n+1}(\vec{y})) \equiv \exists
    \vec{x}^{n+1} \colon F^{n+1}(\vec{y}) \wedge R(F^{0}(\vec{y}),
    \ldots, F^{n}(\vec{y}), \vec{x}^{n+1})
  \]
  is $\Sigma^{(n+1)}_1(M)$. \\
  Note that this transformation only depends on the formulae defining
  $R, F^0, \ldots, F^{n+1}$ and not on $M$, which yields the uniformity we
  claimed.
\end{proof}

\begin{corollary} \label{cor:closure under functions}
  Let $n < \omega$, $1 \le l < \omega$ and
  $R(\vec{x}^{0}, \ldots, \vec{x}^n)$ be a
  $\Sigma^{(n)}_l(M)$-relation. Let $F^0, \ldots, F^{n}$ be such that
  for all $i \le n$ $F^i(\vec{x}^{0}, \ldots, \vec{x}^{(n)})$ is a
  partial $\Sigma^{(i)}_1(M)$-function to $H^i_{M}$. Then
  \[
    R(F^{0}(\vec{x}^{0}, \ldots, \vec{x}^{n}), \ldots,
    F^{n}(\vec{x}^{0}, \ldots, \vec{x}^n))
  \]
  is (uniformly) a $\Sigma^{(n)}_l(M)$-relation.
\end{corollary}

\begin{corollary}
  Let $R(x^{i_{0}}, \ldots, x^{i_l})$ be a
  $\Sigma^{(n)}_l(M)$-relation with $1 \le l < \omega$ and
  $i_0, \ldots, i_k \le n < \omega$. Then there is a
  $\Sigma^{(n)}_l(M)$-relation $R^{*}(x^{0}, \ldots, x^0)$ with the
  same graph as $R$
\end{corollary}

\begin{proof}
  Let $R^{**}$ be the result of replacing each $x^{i_0}$ with $x^{0}$
  in $R$. For $j \le k$ let $F^j$ be the partial
  $\Sigma^{i_j}_1(M)$-function to $H^{i_j}_{M}$ defined by
  \[
    \phi^{j}(x^{0}, y^{i_{l}}) \equiv x^{0} = y^{i_{j}}.
  \]
  Then $R^{**}(F^{i_{0}}(x^{0}), \ldots, F^{i_{k}}(x^{0}))$ is a
  $\Sigma^{(n)}_l(M)$-relation with the same graph as $R$.
\end{proof}

\begin{definition}
  Let $n < \omega$.  The \emph{good $\Sigma^{(n)}_1(M)$-functions}
  consists of the smallest class $\mathcal{G}^{(n)}_1$ such that
  \begin{enumerate}
  \item Every partial $\Sigma^{(i)}_1(M)$-function
    $F(x^{i_{0}}, \ldots, x^{i_{k}})$ to $H^i_M$ with
    $i_0, \ldots, i_k, i \le n$ is in $\mathcal{G}^{(n)}_{1}$ and
  \item $\mathcal{G}^{(n)}_1$ is closed under composition, i.e. if
    $F(x^{i_{0}}, \ldots, x^{i_k})$ is in $\mathcal{G}^{(n)}_1$ and
    $G(\vec{z})$ is a function to $H^{i_j}_M$ for some $j \le k$ in
    $\mathcal{G}^{(n)}_1$, then
    \[ F(x^{i_{0}}, \ldots, x^{i_{j-1}}, G(\vec{z}), x^{i_{j+1}},
      \ldots, x^{i_{k}})
    \] is in $\mathcal{G}^{(n)}_{1}$.
  \end{enumerate}
\end{definition}

\begin{lemma}
  Let $n < \omega, 1 \le l < \omega$, $i_0, \ldots, i_k \le n$ and
  $R(x^{i_{0}}, x^{i_k})$ be a $\Sigma^{(n)}_l(M)$-relation. If, for
  $j \le k$, $F_j(\vec{z})$ is a good $\Sigma^{(n)}_1(M)$-function to
  $H^{i_j}_M$, then
  \[
    R(F_{0}(\vec{z}), \ldots, F_n(\vec{z}))
  \]
  is $\Sigma^{(n)}_l(M)$.
\end{lemma}

\begin{proof}
  Repeated application of \autoref{cor:closure under functions}.
\end{proof}

We are now ready to prove one of the main pillars of basic
$\Sigma^{*}$ fine structure:

\begin{theorem}[$\Sigma^{(n)}_{1}$-Uniformization]
  Let $n < \omega$ and $R(\vec{x}^{0}, \ldots, \vec{x}^{n}, y^n)$ be a
  $\Sigma^{(n)}_1(M)$-relation. Then there is a partial
  $\Sigma^{(n)}_1(M)$-function $F$ to $H^n_M$ such that
  \begin{enumerate}
  \item $\dom(F) = \{ \vec{x} \mid \exists y^{n} R(\vec{x}, y^{n}) \}$ and
  \item
    $\forall \vec{x} ( \exists y^{n} R(\vec{x}, y^{n}) \implies
    R(\vec{x}, F(\vec{x})))$
  \end{enumerate}
  Moreover, $F$ can be chosen to have a uniform definition in $R$'s
  definition.
\end{theorem}

\begin{proof}
  Recall that
  \[
    R_{\vec{x}} := \{ (\vec{x}^{n}, y^{n}) \mid R(\vec{x}^{0}, \ldots,
    \vec{x}^{n}, y^{n}) \}
  \]
  is uniformly
  $\Sigma_1(M^{n, (\vec{x}^{0}, \ldots, \vec{x}^{n-1})})$. Let
  $i < \omega$ be such that $\phi_i$ defines $R_{\vec{x}}$ and let
  \[
    F(\vec{x}^{0}, \ldots, \vec{x}^{n}) := h_{M^{n, (\vec{x}^{0},
        \ldots, \vec{x}^{n-1})}}(i, \vec{x}^{n}).
  \]
  Since $h_{M^{n, (\vec{x}^{0}, \ldots, \vec{x}^{n-1})}}$ is uniformly
  $\Sigma_1(M^{n, (\vec{x}^{0}, \ldots, \vec{x}^{n-1})})$, we have
  that $F$ is $\Sigma^{(n)}_1(M)$ and clearly $F$ uniformizes $R$.
\end{proof}

\begin{remark}
  In these notes, I don't cover the definition of
  $Q^{(n)}$-formulae. Consult Zeman's book.
\end{remark}

\begin{lemma} \label{lemma:Skolem functions}
  Let $n < \omega$. There is a uniform good $\Sigma^{(n)}_{1}(M)$ function
  \[
    F \colon H^{n+1}_{M} \times H^{0}_{M} \to H^{0}_{M}
  \]
  such that for all $r \in R^{n+1}_{M}$
  \[
    F_{r} \colon H^{n+1}_{M} \to M, x^{n+1} \mapsto F(x^{n+1}, p)
  \]
  is surjective.
\end{lemma}

\begin{proof}
  By induction on $n < \omega$.
  \begin{itemize}
  \item[$n=0$ :] Let
    \[
      F \colon H^{1}_{M} \times H^{0}_{M} \to H^{0}_{M}, (w,p) \mapsto
      h_{M}(w(0),(w(1), p(0))).
    \]
  \item[$n \mapsto n+1$ :] Let
    \[
      G \colon H^{n+1}_{M} \times H^{0}_{M} \to H^{0}_{M}
    \]
    be as above. Define
    \[
      F \colon H^{n+1}_{M} \times H^{0}_{M} \to H^{0}_{M}, (w,p)
      \mapsto G(h_{M^{n, p \restriction n}}(w(0), (w(1), p(n))), p \restriction n).
    \]
    Here we let $w(k), p(k), p \restriction k$, $k < \omega$ be the
    usual interpretation for functions if $w,p$ are functions whose
    domain contains $k + 1$ and otherwise we define them to be
    $\emptyset$.
  \end{itemize}
\end{proof}

\begin{exercise}
  Verify that the functions defined above are as desired.
\end{exercise}

Hint: Recall that if $r \in R^{n+1}_M$ then
$r \restriction n \in R^{n}_M$ and
$r(n) \in R_{M^{n, r \restriction n}}$.

\begin{definition}
  For $p \in \Gamma^{n}_{M}$ let
  \begin{enumerate}
    \item $h^{n,p}_{M} := h_{M^{n,p}}$ and
    \item $\tilde{h}^{n}_{M}$ be uniform good
      $\Sigma^{(n-1)}_{1}(M)$-function given nesting Skolem functions
      of the $i$-th projecta as in \autoref{lemma:Skolem functions},
      i.e.
      \[
        \tilde{h}^{1}_{M}(w, p) := h_{M}(w(0), (w(1), p(0)))
      \]
      and
      \[
	\tilde{h}^{n+1}_{M}(w, p) := h^{n}_{M}( h_{M^{n, p
            \restriction n}}(w(0), (w(1), p(n))), p \restriction n).
      \]      
    \end{enumerate}
\end{definition}
\begin{remark}
  $h^{n+1}_{M}$ is a uniformly $\Sigma^{(n)}_{1}(M)$-function and if
  $r \in R^{n+1}_{M}$, then every $x \in M$ is of the form
  $\tilde{h}^{n+1}(z, r)$ for some $z \in H^{n+1}_{M}$. In fact, we
  can choose $z \in \omega \rho^{n+1}_{M}$.
\end{remark}

\begin{corollary} \label{cor:definability via very good parameters}
  Let $1 \le n < \omega$ and $r \in R^{n}_{M}$. Then every
  $A \subseteq H^{n}_{M}$ which is $\boldsymbol{\Sigma}^{n}_{1}(M)$ is
  $\boldsymbol{\Sigma}_{1}(M^{n,r})$.
\end{corollary}

\begin{proof}
  We already know that $\boldsymbol{\Sigma}_1(M^{n,r}) \subseteq \boldsymbol{\Sigma}^{(n)}_{1}(M)$. \\

  Conversely let
  \[
    A = \{ x \in H^{n}_{M} \mid M \models \phi[x, q] \}
  \]
  for some $\Sigma^{(n)}_{1}$-formula $\phi$ and some $q \in M$. Fix
  $\xi < \omega \rho^{n}_{M}$ such that
  $q = \tilde{h}^{n}_{M}(\xi, r)$. Then
  \[
    A(x^{n}) \iff M \models \phi[x^{n}, \tilde{h}^{n}_{M}(\xi^{n}, r^{0})].
  \]
  Since $\tilde{h}^{n}(y^{n}, z^{0})$ is a good
  $\Sigma^{(n-1)}_{1}(M)$-function (and thus a good
  $\Sigma^{(n)}_{1}(M)$-function), this witnesses that $A$ is
  $\Sigma^{(n)}_{1}(M)$ in parameters $\xi, r$ and hence
  $\boldsymbol{\Sigma}_{1}(M^{n,r})$ (by the characterization of
  $\Sigma^{(n)}_{l}(M)$ subsets of $H^{n}_{M}$ as
  $\Sigma_{l}(M^{n,r})$ relations in $\Sigma^{n-1}_{1}(M)$
  predicates).
\end{proof}

\begin{corollary}
  Let $n < \omega$ and $r \in R^{n}_{M}$. Then
  $\rho^{n+1}_{M} = \rho_{M^{n,r}}$.
\end{corollary}

\begin{proof}
  By definition $\rho^{n+1}_{M} \le \rho_{M^{n,r}}$, hence it suffices
  to show the converse. \\
  Let $q \in P^{n+1}_{M}$ and $A$ be $\Sigma^{(n)}_{1}(M)$ in $q$ such
  that $A \cap \omega\rho^{n+1}_{M} \not \in M$. By
  \autoref{cor:definability via very good parameters}, $A$ is
  $\boldsymbol{\Sigma_{1}}(M^{n,r})$, so that
  $\rho_{M^{n,r}} \le \rho^{n+1}_{M}$.
\end{proof}

\begin{corollary}
  Let $r \in R^{n}_{M}$ and $m \le n$. Then
  \begin{itemize}
  \item $\rho^{n}_{M} = \rho^{n-m}_{M^{m, r \restriction m}}$ and
  \item $M^{n, r} = (M^{m, r \restriction m})^{n-m, s}$, where
    $s \colon n-m \to M$ is given by $s(i) := r(m+i)$.
  \end{itemize}
\end{corollary}

\begin{proof}
  By induction on $n-m$ and the two preceeding corollaries.
\end{proof}

\begin{exercise}
  Let $r,s \in R^{n}_{M}$. Then
  \[
    \boldsymbol{\Sigma}_{1}(M^{n,r}) =
    \boldsymbol{\Sigma}_{1}(M^{n,s}).
  \]
\end{exercise}

\begin{remark}
  In these notes, I don't cover functionally absolute definitions of
  good $\Sigma^{(n)}_1(M)$-functions. Consult Zeman's book.
\end{remark}

\begin{lemma} \label{lemma:definability in very good parameters}
  Let $r \in R^{n}_{M}$. Then
  $\boldsymbol{\Sigma}_{l}(M) \subseteq \Sigma^{(n)}_{l}(M)$ for every
  $l \ge 1$.
\end{lemma}

\begin{proof}
  We will prove the result for $l = 1$. The general case follows by a
  straightforward induction. Let $\phi$ be a $\Sigma_{0}$ formula, $q \in M$ and
  \[
    A = \{ a \in M \mid M \models \exists x \phi[x,a,q] \}.
  \]
  Fix $\xi \in \omega \rho^{n}_{M}$ such that
  $q = \tilde{h}^{n}(\xi, r)$. Then
  \[
    A = \{ a \in M \mid M \models \exists x^{(n)} \phi [
    \tilde{h}^{n}(x^{n}, r^{0}), a^{0}, \tilde{h}^{n}(\xi^{n}, r^{0})]
    \}
  \]
  is $\Sigma^{(n)}_{1}(M)$.
\end{proof}

\begin{corollary}
  Let $r \in R^{n}_{M}$. Then
  $\boldsymbol{\Sigma}_{< \omega}(M) = \boldsymbol{\Sigma}^{(n)}_{<
    \omega}(M)$.
\end{corollary}

\begin{proof}
  \autoref{lemma:definability in very good parameters} yields
  $\boldsymbol{\Sigma}_{< \omega}(M) \subseteq
  \boldsymbol{\Sigma}^{(n)}_{< \omega}(M)$. For the converse just note
  that every $\Sigma^{(n)}_{l}$-formulae can be expressed as a
  $\Sigma_{l}$-formula (in parameters) by replacing each occurance of
  the variable $x^{i}$ with $x \in H^{i}_{M}$ if
  $\omega\rho^{i}_{M} < o(M)$ or with $x$ if
  $\omega \rho^{i}_{M} = o(M)$.
\end{proof}

\subsection{Standard Parameters}

\todo[inline]{Talk 2, first draft}
\begin{lemma}
  $h_{M}" [o(M)]^{< \omega} = M$.
\end{lemma}

\begin{proof}
  Let $X := h_{M}" [o(M)]^{<\omega}$. Since $X$ is closed under
  pairing, we have $X \prec_{1} M$. Let
  \[
    N \overset{\pi}{\cong} X \prec_{1} M
  \]
  Be the Mostowski collapse. Since being an acceptable
  $\mathcal{J}$-structure is a $Q$-property, we have that $N$ is an
  acceptable $\mathcal{J}$-structure and clearly
  $\pi^{-1} \restriction o(M) = \id$. Since there is, for acceptable
  $\mathcal{J}$-structures $K$, a uniform lightface
  $\Sigma_{1}$-definable surjection
  $f^{K} \colon [o(M)]^{<\omega} \twoheadrightarrow K$, it follows
  that $N = M$.
\end{proof}

\begin{lemma}
  Let $A$ be $\Sigma^{(n)}_1(M)$ in the parameter $p \in M$. Then
  there is some $p' \in [o(M)]^{<\omega}$ such that $A$ is
  $\Sigma^{(n)}_{1}(M)$ in $p'$.
\end{lemma}

\begin{proof}
  Fix $\phi \in \Sigma^{(n)}_{1}(M)$ such that
  \[
    A = \{ a \in M \mid M \models \phi[a,p] \}.
  \]
  Let $p' \in [o(M)]^{< \omega}$ be such that $p = h_{M}(p')$. Then
  \[
    A = \{ a \in M \mid M \models \exists x^{(0)} \colon
    \underbrace{h_{M}(p') = x^{(0)}}_{\Sigma^{(0)}_{1}} \wedge \phi[a,
    x^{0}] \}.
  \]
  is $\Sigma^{(n)}_{1}(M)$ in $p'$.
\end{proof}

\begin{convention}
  Let $p \in P^{(n)}_{M}$. If, for every $i \in \dom(i)$, $p(i)$ is a
  finite set of ordinals, we will identify $p$ with
  $\bigcup \ran(p) \in [\ord]^{<\omega}$. And we let
  $P^{(n)}_{M} \cap [o(M)]^{<\omega}$ be the collection of these
  good parameters.
\end{convention}
  
By the previous lemma we may restrict ourselves to good parameters
which are finite sets of ordinals, i.e. to
$P^{(n)}_{M} \cap [o(M)]^{<\omega}$ and we shall do so from now on.

\begin{definition}
  Let $a,b \in [\ord]^{<\omega}$.
  \begin{enumerate}
  \item $a \Delta b := (a \setminus b) \cup (b \setminus a)$ and
  \item $a <^{*}b : \iff \max(a \Delta b) \in b$.
  \end{enumerate}
\end{definition}

\begin{exercise}
  Let $a,b \in [\ord]^{<\omega}$. The following are equivalent

  \begin{enumerate}
  \item $a <^{*} b$,
  \item
    $\exists \xi \in b \setminus a \colon b \setminus (\xi + 1) = a
    \setminus (\xi + 1)$,
    \item $\vec{a} <_{\lex} \vec{b}$,
    \end{enumerate}
    where
    $\begin{Bmatrix}
      \vec{a} \\
      \vec{b}
    \end{Bmatrix}$
    is the strictly decreasing enumeration
    of $\begin{Bmatrix}
      a \\
      b
    \end{Bmatrix}$.
  \end{exercise}

  \begin{definition}
    A definable, binary relation $R$ is \emph{set-like} if for all $y$
    \[
      R_{y} := \{ x \mid (x,y) \in R \} 
    \]
    is a set.
  \end{definition}

  \begin{exercise}
    $<^{*}$ is a set-like strict, $\Sigma_{0}$-definable well-order of
    $[\ord]^{< \omega}$.
  \end{exercise}

  (Hint: The only tricky bit is to show that $<^{*}$ is
  well-founded. So, suppose it isn't. Fix a set
  $A \subseteq [\ord]^{<^{*}}$ with no $<^{*}$-minimal
  element. Recursively construct a strictly decreasing sequence
  $(\xi_{n} \mid n < \omega)$ via
  $\xi_{0} := \min \{ \max(a) \mid a \in A \}$ and
  $\xi_{n+1} := \min \{ \max(a \cap \xi_{n}) \mid a \in A \wedge a
  \setminus \xi_{n} = \{ \xi_{0}, \ldots, \xi_{n} \}\}$. Verify that
  this construction never breaks down (i.e. $\xi_{n+1} \neq 0$).)

  \begin{definition}
    The $<^{*}$-least $p \in P^{(n)}_{M} \cap [o(M)]^{< \omega}$,
    denoted by $p_{M,n}$, is called the \emph{$n$-th standard
      parameter of
      $M$}. \footnote{Zeman calls $p_{M,n}$ the standard parameter above $\omega \rho^{n}_{M}$.} \\
    The $<^{*}$-least $p \in P^{*}_{M} \cap [o(M)]^{<\omega}$ is
    called the \emph{(ultimate) standard parameter of $M$}.
  \end{definition}

    \begin{notation}
    \begin{enumerate}
    \item If $a \in [o(M)]^{< \omega}$ we set
      \begin{enumerate}
      \item $a^{n} := a \cap [ \omega\rho^{n+1}_{M}, \omega\rho^{n}_{M}]$ and
      \item $a \restriction n = a \setminus \omega\rho^{n}_{M}$.
      \end{enumerate}
    \end{enumerate}
  \end{notation}
  
  \begin{exercise} \label{exercise: manipulation of standard parameters}
    Let $p \in [o(M)]^{<\omega}$. Then
    \begin{enumerate}
    \item
      $p \in P^{n}_{M} \implies p \restriction (n-1) \in P^{n-1}_{M}$
      and $p^{n-1} \in P_{M^{n-1}, p \restriction n-1}$,
    \item $p \restriction (n-1) \in P^{n-1}_{M}$,
      $p^{n-1} \in P_{M^{n-1}, p \restriction n-1}$ and
      $\omega\rho^{n}_{M} = \omega\rho_{M^{n-1, p \restriction n -1}}
      \implies p \in P^{n}_{M}$.
    \item
      $r \in R^{n}_{M} \iff \forall i < n \colon r^{i} \in R_{M^{i, r
          \restriction i}}$.
    \end{enumerate}

  \end{exercise}

  \begin{corollary}
    $p_{M,n} \setminus \omega\rho^{n}_{M} = \emptyset$.
  \end{corollary}

  \begin{proof}
    $p_{M,n} \setminus \omega\rho^{n}_{M} \le^{*} p_{M,n}$ and by
    \autoref{exercise: manipulation of standard parameters} we have
    that $p_{M,n} \setminus \omega\rho^{n}_{M} \in P^{n}_{M}$.
  \end{proof}

  \begin{corollary} \label{corollary: lengthening standard parameters}
    Let $r \in R^{n}_{M}$. Then $r$ can be lengthended to some
    $p \in P^{n+1}_{M}$. If, in addtion, $r \in [o(M)]^{< \omega}$,
    then $r$ can be lengthened to some
    $p \in P^{n+1}_{M} \cap [o(M)]^{<\omega}$.
  \end{corollary}

  \begin{proof}
    Let $p^{-} \in P_{M^{n, r}}$. Then, by \autoref{exercise:
      manipulation of standard parameters},
    $p := r ^{\frown} p^{-} \in P^{n+1}_{M}$. \\
    If $r \in [o(M)]^{< \omega}$, pick
    $p^{-} \in P_{M^{n, r}} \cap [o(M)]^{< \omega}$ and let
    $p = r \cup p^{-}$.
  \end{proof}

  \begin{corollary}
    Let $M$ be sound and $p \in P^{n}_{M}$. Then $p$ can be lengthened
    to a $p^{*} \in P^{*}_{M}$. If, in addition,
    $p \in [o(M)]^{<\omega}$, then $p$ can be lengthened to some
    $p^{*} \in P^{*}_{M} \cap [o(M)]^{<\omega}$.
  \end{corollary}

  \begin{proof}
    Let $k < \omega$ be such that
    $\omega\rho_{M}^{\omega} = \omega \rho_{M}^{k}$. Now apply
    \autoref{corollary: lengthening standard parameters} $k$
    times.
  \end{proof}

  \begin{corollary}
    Let $M$ be $n$-sound. Then
    $p_{M, n-1} = p_{M,n} \restriction (n-1)$.
  \end{corollary}

  \begin{proof}
    By \autoref{corollary: lengthening standard parameters} there is
    some $q$ such that $p_{M, n-1} \cup q \in P^{n}_{M}$. Note that we
    may pick $q \subseteq [\omega\rho^{n}_{M},
    \omega\rho^{n-1}_{M})$. Now
    $p_{M, n-1} = p_{M, n-1} \setminus \omega\rho^{n-1}_{M}$ and
    $p_{M, n-1} \cup q \le^{*} p_{M,n}$. If
    $p_{M, n-1} \cup q = p_{M,n}$, we are done. Otherwise fix
    $\xi \in p_{M,n} \setminus p_{M, n-1} \cup q$ such that
    \[
      p_{M,n} \setminus (\xi + 1) = (p_{M, n-1} \cup q) \setminus (\xi
      + 1).
    \]
    Since $p_{M,n} \restriction (n-1) \in P^{n}_{M}$ and hence
    $p_{M,n} \restriction (n-1) \le^{*} p_{M, n-1}$, we must have that
    $\xi \ge \omega\rho^{n-1}$ and thus
    $p_{M, n-1} = p_{M,n} \restriction (n-1)$ as desired.
  \end{proof}
  
  We are now ready to prove the main result of this section

  \begin{theorem} \label{theorem: characterization of n-soundness}
    $M$ is $n$-sound iff $p_{M,n} \in R^{n}_{M}$.
  \end{theorem}

  \begin{proof}
    If $M$ is $n$-sound, then $p_{M,n} \in P^{n}_{M} = R^{n}_{M}$ and there is nothing to do. \\
    Conversely, suppose that $n$ is minimal with
    $p_{M,n} \in R^{n}_{M}$ but $M$ is not $n$-sound. We will derive a
    contradiction via the downward extension of embeddings lemma: \\

    Let $q = \min_{<^{*}} (P^{n}_{M} \setminus R^{n}_{M})$. Since
    $p_{M,n} = \min_{<^{*}} P^{n}_{M}$, we have $p_{M,n} <^{*} q$. Let
    $i < \omega$ be minimal such that
    $(p_{M,n} \cap [\omega\rho^{i+1}_{M}, \omega\rho^{i}_{M})
    =)p_{M,n}^{i} <^{*} q^{i}$.

    \begin{claim}
      $i = n-1$.
      
    \end{claim}

    \begin{proof}
      Suppose $i < n-1$. Consider the map
      \[
	\id \colon M^{n, q} \to M^{n, q}.
      \]
      By the downward extension of embeddings lemma there are unique
      $\pi, \bar{M}, \bar{q}$ such that
      \begin{enumerate}
      \item $\pi \colon \bar{M} \to_{\Sigma^{(n)}_{1}} M$,
      \item $\pi \restriction H^{n}_{M} = \id$,
      \item $\pi(\bar{q}) = q$,
      \item $\bar{q} \in R^{n}_{\bar{M}}$ and
      \item $\bar{M}^{n, \bar{q}} = M^{n,q}$.
      \end{enumerate}
      By induction hypothesis
      $p_{M,n} \restriction (i+1) = p_{M, i+1} \in R^{i+1}_{M}$. We
      may thus fix a good $\Sigma^{(i)}_{1}(M)$-function and some
      $z \in [\omega\rho^{i+1}_{M}]$ such that
      $q^{i} = f(z, p_{M,n} \restriction i, p_{M,n}^{i})$. Since
      $p_{M,n} \restriction i = q \restriction i$, this witnesses
      \[
	M \models \exists z^{i+1} \exists r^{i} <^{*} q^{i} \colon
        z^{i+1}, r^{i} \in [\ord]^{<\omega} \wedge q^{i} = f(z^{i+1},
        q \restriction i, r^{i}).
      \]
      This is a $\Sigma^{(i+1)}_{1}$-statement and thus preserved
      downwards by $\pi$. Hence there is a finite set of ordinals
      $\bar{z} \subseteq \omega\rho^{i+1}_{\bar{M}}$,
      $\bar{r} \subseteq \omega\rho^{i+1}_{\bar{M}}$ such that
      $\bar{q}^{i} = \bar{f}(\bar{z}, \bar{q} \restriction i
      ,\bar{r})$, where $\bar{f}$ is the interpretation of $f$'s
      $\Sigma^{(i)}_{1}$ definition over $\bar{M}$.  Let
      $z = \pi(\bar{z})$, $r = \pi(\bar{r})$. Then
      $q^{i} = f(z, q \restriction i, r)$. Now consider
      \[
	q^{*} := q \restriction i \cup r \cup z \cup (q \cap
        \omega\rho^{i+1}_{M}).
      \]
      $q$ and $q^{*}$ can be translated into each other via a
      $\Sigma^{(i)}_{1}(M)$ function, so that
      $q^{*} \in P^{n}_{M} \restriction R^{n}_{M}$. On the other hand,
      sincee $r < q^{i}$, $q^{*} <^{*} q$. This contradicts the fact
      that $q = \min_{<^{*}} P^{n}_{M} \setminus R^{n}_{M}$.
    \end{proof}
    We may now run the same proof as for the claim, but for $i = n-1$,
    to show that in fact $R^{n}_{M} = P^{n}_{M}$.
  \end{proof}

  \subsection{Two Applications to $L$}

  \begin{theorem} \label{theorem: L is sound}
    For every $\alpha \in \ord$ $J_{\alpha}$ is acceptable and sound.
  \end{theorem}

  Before we can prove this, we need one more lemma about
  $\Sigma^{(n)}_{1}$-definability:

  \begin{lemma} \label{lemma: Sigma n+1 subsets of the n-th reduct}
    Suppose $R^{n}_{M} \neq \emptyset$. If
    $A \in \boldsymbol{\Sigma}_{n+1}(M) \cap \mathcal{P}(H^{n}_{M})$,
    then $A$ is $\boldsymbol{\Sigma}^{(n)}_{1}(M)$.
  \end{lemma}

  \begin{proof}
    For $n = 0$ this is trivial. We'll provide the proof for $n = 1$
    and leave the induction step as an exercise. \\
    Let $\phi$ be a $\Sigma_{0}$-formula, $\vec{p} \in M$ such that
    \[
      A = \{ a \mid M \models \exists x \phi[a,x,\vec{p}] \}.
    \]
    Now let $r \in R^{1}_{M}$ and fix $\xi \in \omega\rho^{1}_{M}$
    such that $\vec{p} = h_{M}(\xi, r)$. We have
    \begin{align*}
      A & = \{ a \mid M \models \exists x^{1} \phi[a^{1}, h_{M}(x^{1}, r^{0}), h_{M}(\xi^{1}, r^{0})] \} \\
      &= \{ a \mid M^{1, r} \models \exists x \phi^{*}[a, x, \xi] \}
    \end{align*}
    where $phi^{*}$ is the natural $\Sigma_{1}$-formula corresponding
    to $\phi$. \footnote{Recall that $\phi$ is $\Sigma_{1}$ in the
      language $\{ \in, A^{1, r}_{M} \}$} Since $A$ is
    $\Sigma_{1}(M^{1,r})$, it is $\Sigma^{(1)}_{1}(M)$.
  \end{proof}

  \begin{exercise}
    Provide the induction step.
  \end{exercise}

  (Hint: Let $r \in R^{(n+1)}_{M}$. Then $r(0) \in R^{1}_{M}$ and
  $H^{n+1}_{M} = H^{n}_{M^{1,r(0)}}$. Furthermore recall that
  $M^{r(0)}$ has a very good parameter $r^{*} \in R^{n}_{M^{1,r(0)}}$
  such that $r = r(0) ^{\frown} r^{*}$. Use the induction hypothesis
  on $M^{1, r(0)}$.)

  \begin{proof}[Proof of \autoref{theorem: L is sound}]
    We proceed by induction on $\alpha$: \\
    First let $\alpha = 1$: $J_{1} = (V_{\omega}; \in)$ is trivially
    acceptable and since $V_{\omega}$ is the image of $\omega$ under
    the Ackermann coding function (without parameter), it is also
    sound.

    Suppose $J_{\beta}$ is acceptable and sound for all
    $\beta < \alpha$. We will first show that $J_{\alpha}$ is
    acceptable. If $\alpha$ is a limit ordinal, there is nothing to
    do. So suppose that $\alpha = \beta + 1$. It suffices
    to show that
    \[ (\exists \tau < \beta \exists a \subseteq \tau a \in
      J_{\beta+1} \setminus J_{\beta}) \implies \exists f \in
      J_{\beta + 1} f \colon \tau \twoheadrightarrow J_{\beta}.
    \]
    Fix $a, \tau$ as above with $\tau$ minimal.
    \begin{claim}
      $\tau = \omega \rho^{\omega}_{J_{\beta}}$.
      \end{claim}

      \begin{proof}
        Let $n < \omega$ be such that
        $\omega \rho^{\omega}_{J_{\beta}} = \omega
        \rho^{n}_{J_{\beta}}$. Then there is some
        $\boldsymbol{\Sigma}^{(n-1)}_{1}(M)$-subset of
        $\omega \rho^{\omega}_{J_{\beta}}$ not in $J_{\beta}$. Since
        $\boldsymbol{\Sigma}_{<\omega}(J_{\beta}) \subseteq
        J_{\beta+1}$, this new subset is in $J_{\beta+1}$ and hence
        witnesses that $\tau \leq \omega\rho^{\omega}_{J_{\beta}}$. \\
        Conversely, let $a \subseteq \tau$ be
        $\boldsymbol{\Sigma}_{n}(J_{\beta})$. Since
        $\tau \le \omega\rho^{n}_{J_{\beta}}$, we have that
        $a \subseteq H^{n}_{J_{\beta}}$. Since $J_{\beta}$ is sound,
        we have $R^{n}_{J_{\beta}} \neq \emptyset$. And hence, by
        \autoref{lemma: Sigma n+1 subsets of the n-th reduct}, we have
        that $a$ is $\boldsymbol{\Sigma}^{(n-1)}_{1}(J_{\beta})$,
        witnessing that $\omega\rho^{\omega}_{J_{\beta}} \le \tau$.
      \end{proof}
      Now, since $J_{\beta}$ is sound, there is a
      $\boldsymbol{\Sigma}^{*}(J_{\beta})$ function
      \[
        f \colon \omega\rho^{\omega}_{J_{\beta}} \twoheadrightarrow
        J_{\beta}.
      \]
      But $f \in \boldsymbol{\Sigma}_{<\omega}(J_{\beta})
      \subseteq J_{\beta+1}$. Thus $J_{\beta+1}$ is acceptable. \\

      Let us now verify that $J_{\alpha}$ is sound: \\

      By \autoref{theorem: characterization of n-soundness}, it
      suffices to show that $p_{n, J_{\alpha}} \in R^{n}_{J_{\alpha}}$
      for all $n < \omega$. Suppose this is false and let $n < \omega$
      be minimal such that
      $p :=p_{n,J_{\alpha}} \not \in R^{n}_{J_{\alpha}}$. Let $a$ be
      $\Sigma^{(n-1)}_{1}(M)$ in $p$ such that
      $a \cap \omega\rho^{n}_{J_{\alpha}} \not \in J_{\alpha}$. Once
      again, consider
      \[
	\id \colon (H^{n}_{J_{\alpha}}; \in, A^{n, p}_{J_{\alpha}}) \to
        (H^{n}_{J_{\alpha}}; \in, A^{n, p}_{J_{\alpha}}).
      \]
      By the downward extension of embeddings lemma there are unique
      $\pi, \bar{M}, \bar{p}$ such that
      \begin{enumerate}
      \item $\pi \colon \bar{M} \to_{\Sigma^{(n)}_{1}} J_{\alpha}$,
      \item $\pi \restriction H^{n}_{M} = \id$,
      \item $\pi(\bar{p}) = p$,
      \item $\bar{p} \in R^{n}_{\bar{M}}$ and
      \item
        $\bar{M}^{n, \bar{q}} = (H^{n}_{J_{\alpha}}; \in, A^{n,
          p}_{J_{\alpha}})$.
      \end{enumerate}
      By condensation we have that $\bar{M} = J_{\bar{\alpha}}$ for
      some $\bar{\alpha} \leq \alpha$.

      \begin{claim}
        $\bar{\alpha} = \alpha$.
      \end{claim}

      \begin{proofblack}
        Let $\bar{a}$ be the $\Sigma^{(n-1)}_{1}(J_{\bar{\alpha}})$
        set in $q$ via the definition for $a$, call it $\phi$. Let
        $\xi < a \cap \omega\rho^{n}_{J_{\alpha}} =
        \omega\rho^{n}_{J_{\bar{\alpha}}}$. We have
        \begin{align*}
          \xi \in \bar{a} & \iff J_{\bar{\alpha}} \models \phi[\xi, q] \\
                          & \iff J_{\alpha} \models \phi[\underbrace{\pi(\xi)}_{= \xi}, \underbrace{\pi(q)}_{= p}] \\
                          & \iff \xi \in a,
        \end{align*}
        i.e.
        $a \cap \omega\rho^{n}_{J_{\alpha}} = \bar{a} \cap
        \omega\rho^{n}_{J_{\alpha}}$. If $\bar{\alpha} < \alpha$, then
        $\bar{a} \cap \omega\rho^{n}_{J_{\alpha}} \in
        J_{\bar{\alpha}+1} \subseteq J_{\alpha}$, which is
        absurd. Hence $\bar{\alpha} = \alpha$.
      \end{proofblack}

      Since $\pi(q) = p$, we have $q \le^{*} p$. On the other hand $p$
      is the $<^{*}$-least good $n$th parameter and thus $p = q$. By
      $q \in R^{n}_{J_{\bar{\alpha}}} =
      R^{n}_{J_{\alpha}}$. Contradiction!
    \end{proof}


    \section{More on Downward Extensions of Embeddings}
    \todo[inline]{Talk 3, first draft}
    
    \begin{definition}
      Let $M = (|M|; A_{1}, \ldots, A_{n})$ and $X \subseteq |M|$. Then
      \[
	M | X := (|M| \cap X; A_{1} \cap X, \ldots, A_{n} \cap X).
      \]
    \end{definition}

    \begin{lemma}
      Let $M = (J^{A}_{\alpha}; \in, A, B)$ be an acceptable structure
      and let $X \subseteq M$ be closed under good
      $\Sigma^{(n)}_{1}(M)$-functions with
      $X \cap P^{n}_{M} \neq \emptyset$. Let $\bar{M}$ be the
      transitive collapse of $X$ and let $\sigma$ be the inverse of
      the collapsing map. Then
      \[
	\sigma \colon \bar{M} \to_{\Sigma^{(n)}_{1}} M.
      \]
    \end{lemma}

    \begin{proof}
      $\vec{x} \in X$ and let $\phi$ be a $\Sigma^{(n)}_{0}$-formula
      such that
      \[
	M \models \exists y^{n} \phi[y^{n}, \vec{x}].
      \]
      By $\Sigma^{(n)}_{1}(M)$-uniformization, leveraging that $X$ is
      closed under good $\Sigma^{(n)}_{1}(M)$-functions, we obtain
      that
      \[
	M | X \models \exists y^{n} \phi[y^{n}, \vec{x}]
      \]
      and thus that $M | X \prec_{\Sigma^{(n)}_{1}} M$. Now, for
      $i \le n$, define
      \begin{enumerate}
      \item $H^{i} := \sigma " H^{i}_{M}$ and
      \item $\omega\rho^{i} := H^{i} \cap \ord$.
      \end{enumerate}
      We establish a pseudo-interpretation of
      $\Sigma^{(n)}_{1}$-formulae over $\bar{M}$ by declaring that
      variables of type $i$ ($i \le n$) range over $H^{i}$. For this
      pseudo-interpretation we obtain
      \[
	\sigma \colon \bar{M} \to_{\Sigma^{(n)}_{1}} M
      \]
      as above. It now suffices to show
      \begin{claim}
        $\omega\rho^{i}_{\bar{M}} = \omega\rho^{i}$ for all $i \le n$.
      \end{claim}

      \begin{proofblack}
        \todo[inline]{provide proof}
      \end{proofblack}
    \end{proof}

    \begin{lemma}
      Let
      \[
	\sigma \colon \bar{M} \to_{\Sigma^{(n)}_{1}} M
      \]
      be such that $\sigma \restriction \omega\rho^{n+1}_{M} = id$ and
      $\ran(\sigma) \cap P^{*}_{M} \neq \emptyset$. Then $\sigma$ is
      $\Sigma^{*} (= \bigcup_{k <
        \omega}\Sigma^{(k)}_{1})$-elementary.
    \end{lemma}

    \begin{proof}
      \todo[inline]{provide proof}
    \end{proof}

    \section{Witnesses and Solidity}

    \begin{problem}
      Let
      \[
	\sigma \colon \bar{M} \to_{\Sigma^{(n)}_{1}} M \footnote{In a typical application, $\sigma$ will be the
        canonical embedding associated to a fine structural ultrapower
        of $\bar{M}$.}
      \]
      
      In general we might have that
      \[
	p_{M, n+1} <^{*} \sigma(p_{\bar{M}, n+1}),
      \]
      i.e. that standard parameters are moved incorrectly by fine
      structural maps. In this section we introduce the concept of
      solidity that, provided it holds for $\bar{M}$, implies that
      standard parameters will be moved correctly.
    \end{problem}

    \begin{remark}
      Fine structure theory and its later refinement
      $\Sigma^{(*)}$-theory was developed by Ronald Jensen. The
      concept of solidity, however, is due to Bill Mitchell and later,
      independently, Sy Friedman.
    \end{remark}

    \section*{Bonus Exercises}

    \begin{exercise}
      Let $\pi \colon M \to N$ be $\Sigma^{*}$-elementary such that
      $P^{*}_{N} \cap \ran(\pi) \neq \emptyset$. Then, for all $n < \omega$, either
      \begin{enumerate}
      \item $\omega\rho^{n}_{M} = o(M)$ and $\omega\rho^{n}_{N} = o(N)$ or
      \item $\omega\rho^{n}_{M} < o(M)$ and
        $\pi(\omega\rho^{n}_{M}) = \omega\rho^{n}_{N}$.
      \end{enumerate}
      
    \end{exercise}
    
    % \begin{proof}
    %   Let us adapt the convention, for this proof only, that
    %   $\pi(o(M)) = o(N)$. \\
    %   We proceed by induction on $n < \omega$. For $n = 0$ we have
    %   $\omega\rho^{n}_{M} = o(M)$ and $\omega\rho^{n}_{N} = o(N)$. \\
    %   Now assume that the claim holds for all $n$. \\
    %   First we show that
    %   $\omega\rho^{n+1}_{N} \le \pi(\omega\rho^{+1n}_{M})$. If
    %   $\omega\rho^{n}_{M} = \omega\rho^{n+1}_{M}$ then
    %   $\omega\rho^{n+1}_{N} \le \omega\rho^{n}_{N} =
    %   \pi(\omega\rho^{n}_{M}) = \pi(\omega\rho^{n+1}_{M})$. Thus we
    %   may assume that $\omega\rho^{n+1}_{M} < \omega\rho^{n}_{M}$. \\
    %   Let $\phi$ be a $\Sigma^{(n)}_{1}$ formula and $p \in M$ such
    %   that
    %   \[
    %     \{ \xi < \omega\rho^{n+1}_{M} \mid M \models \phi[\xi, p] \}
    %     \not \in M
    %   \]
    %   Since $\omega\rho^{n+1}_{M} < \omega\rho^{n}_{M} \le o(M)$ and
    %   $M$ is acceptable, it follows that
    %   \[
    %     \{ \xi < \omega\rho^{n+1}_{M} \mid M \models \phi[\xi, p] \}
    %     \not \in H^{n}_{M}
    %   \]
    %   Consider
    %   \[
    %     A := \{ \xi < o(N) \mid N \models \phi[\xi, \pi(p)] \}.
    %   \]
    %   It suffices to show that
    %   $A \cap \pi(\omega\rho^{n+1}_{M}) \not \in N$. \\
    %   Suppose $A \cap \pi(\omega\rho^{n+1}_{M}) \in N$. Since
    %   $\omega\rho^{n+1}_{M} < \omega\rho^{n}_{M}$, we have
    %   $\pi(\omega\rho^{n+1}_{M}) < \pi(\omega\rho^{n}_{M}) =
    %   \omega\rho^{n}_{N}$. Hence $A \cap \pi(\omega\rho^{n+1}_{M})$ is
    %   a bounded subset of $H^{n}_{N}$ and thus, by acceptability and
    %   the fact that either $\omega\rho^{n}_{N} = o(N)$ or
    %   $\omega\rho^{n}_{N}$ is an $N$-cardinal, in $H^{n}_{N}$. It
    %   follows that
    %   \[
    %     N \models \exists x^{n} \forall \xi \in
    %     \pi(\omega\rho^{n+1}_{M}) \xi \in x^{n} \iff \phi[\xi,
    %     \pi(p)].
    %   \]
    %   This is a $\Sigma^{(n)}_{1}$-statement and hence preserved downwards, i.e.
    %   \[
    %     M \models \exists x^{n} \forall \xi \in
    %     \omega\rho^{n+1}_{M} \xi \in x^{n} \iff \phi[\xi,
    %     \pi].
    %   \]
    %   But this witnesses that
    %   \[
    %     \{ \xi < \omega\rho^{n+1}_{M} \mid M \models \phi[\xi, p] \}
    %    \in H^{n}_{M},
    %   \]
    %   Contradiction! Hence we do have that
    %   $\omega\rho^{n+1}_{N} \le \pi(\omega\rho^{n+1}_{M})$. \\

    %   Conversely let $p \in M$ be such that $\pi(p) \in P^{*}_{N}$ and
    %   let $\phi$ be a $\Sigma^{(n)}_{1}$-formula such that
    %   \[
    %     \{ \xi < \omega\rho^{n+1}_{N} \mid N \models \phi[\xi, \pi(p)] \} \not \in N
    %   \]
    %   Suppose that $\pi(\omega\rho^{n+1}_{M}) < \omega\rho^{n+1}_{N}$. Then
    %   \[
    %     \{ \xi < \pi(\omega\rho^{n+1}_{M})\mid N \models \phi[\xi,
    %     \pi(p)] \} \in N
    %   \]
    %   However, by the proof of Lemma 1.7.5 in Zeman's book we have
    %   that
    %   \[
    %     \bar{A} := \{ \xi < \omega\rho^{n+1}_{M} \mid M \models \phi[\xi, p] \}
    %     \in M.
    %   \]
    %   By $\Sigma^{(n+1)}_{1}$-elementarity we have that
    %   \[
    %     N \models \forall \xi^{n+1} \phi[\xi^{n+1}, \pi(p)] \iff
    %     \xi^{n+1} \in \pi(\bar{A}),
    %   \]
    %   i.e.
    %   \[
    %     \{ \xi < \omega\rho^{n+1}_{N} \mid N \models \phi[\xi, \pi(p)]
    %     \} = \pi(\bar{A}) \in N.
    %   \]
    %   Contradiction!      
    % \end{proof}
    
\end{document}

%%% Local Variables:
%%% mode: latex
%%% TeX-master: t
%%% End:
